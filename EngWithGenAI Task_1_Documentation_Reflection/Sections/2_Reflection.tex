\section{Reflection}
\label{sec:reflection}

Reflect on the challenges encountered during the project, the solutions devised, and potential areas for future improvement, including a brief section on ethical considerations and data privacy best practices when integrating external data sources could be beneficial. The purpose of the reflection is to show that you are able to reflect and assess your own work and learning process critically. It should answer the following questions:

\begin{enumerate}
    \item What was the most interesting thing that you learned while working on the portfolio? What aspects did you find interesting or surprising?
    \item Which part of the portfolio are you (most) proud of? Why? What were the challenges you faced and how did you overcome them?
    \item What adjustments to your design and implementation were necessary during the implementation phase? What would you change or do differently if you had to do the portfolio task a second time? What would be potential areas for future improvement. 

    \item Include a brief section on ethical considerations when using these models on code generation tasks. 
    \item From the lecture/course including guest lectures, what topic excited you the most? Why? What would you like to learn more about and why?

    \item Based on the content of the lecture taught during the semester and the task you have carried out in this portfolio, describe a project that you would like to do using generative AI and LLMs. 

    \item What would be a good multiple-choice question for a test to see if a student has really understood the content?  Design 3 Multiple Choice Questions (with at least 3 answer choices) keeping the following in mind:

    \begin{enumerate}
        \item The question and answers to it critically tests a student’s understanding of the content given in the course, however, the questions should not be too general or straightforward to answer and should have an objective answer.  
        \item Formulate the questions clearly such that minimal reading effort is required. Keep the questions challenging to answer but make sure that one answer is clearly the correct or best option. 
        \item Take inspiration from the MCQ test that you had taken on 30.01.2024 to formulate your questions. For specific instructions please take a careful look at the document “Rules\_For\_Creating\_MCQs.pdf” in your respective task’s folder in “Downloads > Portfolio Exam”
        \item Use of LLMs is strictly prohibited.

    \end{enumerate}
        
    
\end{enumerate}




 

 

